\chapter{DESAIN}

Pada bab ini akan dijelaskan desain algoritma yang akan digunakan untuk menyelesaikan permasalahan.

\section{Deskripsi Umum Sistem}

Sistem akan menerima tujuh nilai masukan: $ N $, $ L $, $ q $, $ p $, $ g $, $ y $, $ Hm $. Nilai masukan ini mengikuti penjelasan pada subbab \ref{sec:Parameter Masukan}. Kemudian sistem akan mencari sebuah nilai $ x $ pada persamaan $ y=g^x\ (mod\ p) $. Nilai $ x $ ini akan digunakan pada tahapan pembuatan \textit{signature}, mengacu pada prosedur yang dijabarkan pada subbab \ref{sec:Keluaran Permasalahan}. Seusai pembuatan \textit{signature}, sistem akan mengeluarkan \textit{signature} yang telah dibuat yaitu sepasang nilai \textit{(r, s)}.

Pada tahap penyelesaian logaritma diskret, sistem menggunakan dua metode yang berbeda. Masing-masing metode akan digunakan untuk dibandingkan kinerja satu sama lainnya.

\section{Desain Program Utama}

Subbab ini akan menjelaskan desain fungsi yang akan digunakan untuk menyelesaikan permasalahan.

\subsection {Desain Fungsi Main}

Pada fungsi ini, semua tahapan komputasi berjalan. Termasuk pada tahapan ini yaitu pencarian logaritma diskret dan pembuatan \textit{signature}. Fungsi ini menerima parameter secara interaktif dari pengguna, dan mengeluarkan hasil \textit{signature} \textit{(r, s)} secara interaktif kepada pengguna. Desain fungsi ini dapat dilihat pada gambar \ref{psdo:main}.
\begin{figure}[h!]
\begin{lstlisting}[firstnumber=0]
MAIN
let n, l, q, p, g, y, Hm = Input()
let x = DISCRETE-LOG (q, p, g, y)
let r, s = SIGN (q, p, g, Hm, x)
output (r, s)
\end{lstlisting}
\caption{Desain Fungsi \textit{Main}}
\label{psdo:main}
\end{figure}

\subsection{Desain Fungsi Pencarian Logaritma Diskret}

Terdapat dua metode pencarian logaritma diskret yang akan diujikan pada penyelesaian permasalahan. Subbab ini akan menjelaskan desain masing-masing metode.

\subsubsection {Desain Fungsi Baby-step Giant-step}

Subbab ini akan menjabarkan desain fungsi \textit{Baby-step Giant-step} menurut penjelasan pada subbab \ref{sec:Strategi Penyelesaian Logaritma Diskret dengan Baby-step Giant-step} dalam bentuk \textit{pseudocode}. Fungsi ini menerima tiga \textit{integer}, yaitu $ g $, $ y $, $ n $. Keluaran fungsi ini adalah sebuah \textit{integer} $ x $. Masukan dan keluaran fungsi ini mengacu pada persamaan \eqref{eq:persamaan_umum_log_diskret}. Desain fungsi ini dapat dilihat pada gambar \ref{psdo:bsgs}.
\begin{figure}[h!]
\begin{lstlisting}[firstnumber=0]
BABY-STEP-GIANT-STEP (g, y, n)
let limit = sqrt (ord(g)) rounded upward
let GIANTSTEPS[] be a new array
let BABYSTEPS[] be a new array
for i = 0 to limit-1
	store pair <s = limit*i, t = g ^ (limit*i)> into GIANTSTEPS
for j = 0 to limit-1
	store pair <s = j, t = y * g^j> into BABYSTEPS
find a match between GIANTSTEPS and BABYSTEPS on the t.
if a pair (x, y) such that said match exist
	return GIANTSTEPS[x].s – BABYSTEPS[y].s
return "NOT FOUND"
\end{lstlisting}
\caption{Desain Fungsi \textit{Baby-step Giant-step}}
\label{psdo:bsgs}
\end{figure}

\subsubsection {Desain Fungsi Pollard Rho dengan Brent Cycle Detection}

Subbab ini akan menjabarkan desain fungsi \textit{Pollard Rho} dengan \textit{Brent Cycle Detection} menurut penjelasan pada subbab \ref{sec:Strategi Penyelesaian Logaritma Diskret dengan Pollard Rho} dan \ref{sec:Brent Cycle Detection}. Fungsi ini menerima tiga \textit{integer}, yaitu $ g $, $ y $, dan $ n $. Keluaran fungsi ini adalah \textit{integer} $ x $. Masukan dan keluaran fungsi ini mengacu pada persamaan \eqref{eq:persamaan_umum_log_diskret}. Desain fungsi ini dapat dilihat pada gambar \ref{psdo:brent_pollard_rho}.
\begin{figure}[h!]
\begin{lstlisting}[firstnumber=0]
POLLARD-RHO-BRENT (g, y, n)
let p1 = 1, p2 = 1
let i = 2
let iteration = 0
do
	move p1 forward once 
	iteration = iteration + 1
	if iteration == 2^i
		move p2 to p1
		i = i + 1
		iteration = 0
while p1 != p2
let inverse = (p1.b – p2.b)-1 (mod n)
return (p2.a – p1.a) * inverse (mod n)
\end{lstlisting}
\caption{Desain Fungsi \textit{Pollard Rho} dengan \textit{Brent Cycle Detection}}
\label{psdo:brent_pollard_rho}
\end{figure}

\subsubsection {Desain Fungsi Step Untuk Pollard Rho}
Subbab ini akan menjabarkan desain fungsi \textit{Step} pada \textit{Pollard Rho}. Pada subbab \ref{sec:Strategi Penyelesaian Logaritma Diskret dengan Pollard Rho}. dijelaskan bahwa fungsi step butuh mempartisikan himpunan $ \mathbb{Z}_p $ menjadi tiga himpunan. Partisi ini dapat dilakukan dengan mudah dengan melakukan modulus setiap elemen $ \mathbb{Z}_p $ dengan 3, sehingga terbentuk tiga kelas residu: $ [0]_3,[1]_3,[2]_3 $. Ketiga kelas residu ini merepresentasikan tiga himpunan yang digunakan fungsi \textit{}step, dimana $ S_1=[1]_3 $, $ S_2=[0]_3 $, dan $ S_3=[2]_3 $.

Fungsi ini menerima empat \textit{integer} sebagai parameter. Keluaran fungsi ini adalah sebuah \textit{integer} hasil proses \textit{step}. Desain fungsi ini dapat dilihat pada gambar \ref{psdo:step}.
\begin{figure}[h!]
\begin{lstlisting}[firstnumber=0]
STEP (value, alpha, beta, modulus)
let class = value mod 3
if class == 0
	value = MODULAR-MULTIPLICATION (value, value, modulus)
else if class == 1
	value = MODULAR-MULTIPLICATION (value, beta, modulus)
else
	value = MODULAR-MULTIPLICATION (value, alpha, modulus)
return value
\end{lstlisting}
\caption{Desain Fungsi \textit{Step}}
\label{psdo:step}
\end{figure}

\subsection{Desain Fungsi Sign}

Subbab ini akan menjabarkan desain fungsi Sign yang mengikuti prosedur pada subbab \ref{sec:Keluaran Permasalahan}. Fungsi ini menerima lima \textit{integer} sebagai parameter dan dua \textit{integer} sebagai keluaran. Desain fungsi ini dapat dilihat pada gambar \ref{psdo:sign}.
\begin{figure}[h!]
\begin{lstlisting}[firstnumber=0]
SIGN (globalMod, keyMod, g, hash, x)
let i = 1
let r, s = 0, 0

while r == 0 do
	i = i + 1
	r = MODULAR-EXPONENTIATION (g, i, keyMod, globalMod)
	if r != 0
		let inverse = MODULAR-INVERSE (generator, keyMod)
		s = (hash + r * x) mod q
		s = (s * inverse) mod q
		if s != 0
			return (r, s)
	r = 0
\end{lstlisting}
\caption{Desain Fungsi \textit{Sign}}
\label{psdo:sign}
\end{figure}

\subsection{Desain Fungsi Modular Exponentiation}

Subbab ini akan menjabarkan desain fungsi \textit{Modular Exponentiation}, mengacu pada penjelasan pada subbab \ref{sec:Strategi Penyelesaian Pemangkatan Modular dengan Repeated Squaring}. Fungsi ini menghitung $ g^k\ mod\ p $, dimana $ g $ memiliki order elemen $ q $. Fungsi ini menerima empat \textit{integer} sebagai input dan mengeluarkan sebuah \textit{integer} sebagai hasil penghitungan pemangkatan modular. Desain fungsi ini dapat dilihat pada gambar \ref{psdo:modex}.
\begin{figure}[h!]
\begin{lstlisting}[firstnumber=0]
MODULAR-EXPONENTIATION (g, k, p, q)
let result = 1
let basis = 1
k = k mod q
let ~$ \left\langle k_0,\ k_1,\ \dots,\ k_n\right\rangle $~ be a bit representation of k where ki is the ith bit.
for i = 1 to n
	basis = (basis * g) mod p
		if ~$ k_i $~ == 1
		result = (result * basis) mod p
return result
\end{lstlisting}
\caption{Desain Fungsi \textit{Modular Exponentiation}}
\label{psdo:modex}
\end{figure}

\subsection{Desain Fungsi Logarithmic Modular Multiplication}

Subbab ini akan menjabarkan desain fungsi \textit{Logarithmic Modular Multiplication} mengikuti penjelasan pada subbab \ref{sec:Strategi Perkalian Modular dengan Logarithmic Modular Multiplication}. Fungsi ini menerima tiga masukan: sebuah nilai yang dikali $ a $, sebuah nilai pengali $ b $, dan sebuah modulus $ n $. Keluaran fungsi ini adalah sebuah \textit{integer} hasil perkalian. Desain fungsi ini dapat dilihat pada gambar \ref{psdo:modmul}.
\begin{figure}[h!]
\begin{lstlisting}[firstnumber=0]
LOGARITHMIC-MODULAR-MULTIPLICATION (a, b, n)
let result = 1, basis = a
let ~$ \left\langle b_1,\ b_2,\ \dots ,\ b_t \right\rangle $~ be the i-th bit representation for b
for i = 1 to t
	if ~$ b_i $~ == 1
		result = (result + basis) mod n
	basis = (basis * 2) mod n
return result
\end{lstlisting}
\caption{Desain Fungsi Modular Multiplication}
\label{psdo:modmul}
\end{figure}

\subsection{Desain Fungsi Invers Modulus}

Fungsi ini digunakan pada banyak bagian penyelesaian permasalahan. Cara kerja fungsi ini mengacu pada penjelasan di subbab \ref{sec:Strategi Penyelesaian Invers Modulus dengan Extended Euclidean}. Fungsi menerima sebuah persamaan \ref{eq:persamaan_umum_pembagian}. Keluaran metode ini adalah koefisien sebuah nilai $ a $ dan sebuah modulus $ b $. Desain fungsi ini dapat dilihat pada gambar \ref{psdo:modinv}.
\begin{figure}[h!]
\begin{lstlisting}[firstnumber=0]
MODULAR-INVERSE (a = qb + r)
let EQUATION[] be a new array
let i = 0
while r != 0
	EQUATION[i] = (a = qb + r)
	find equation b = qr + m for some q and m
	i = i + 1
for i = EQUATION.len – 2 downto 0
	substitute EQUATION[i].b with EQUATION[i+1].r
return (EQUATION[0].a.coefficient, EQUATION[0].b.coefficient)
\end{lstlisting}
\caption{Desain Fungsi Invers Modulus}
\label{psdo:modinv}
\end{figure}

\subsection{Desain Fungsi Uji Kebenaran}

Fungsi uji kebenaran digunakan untuk melakukan uji coba kebenaran keseluruhan program pada tahapan komputasi utama. Fungsi ini akan digunakan sebagai \textit{catch-all error detection} untuk membantu proses \textit{debugging}. Fungsi ini bekerja mengikuti prosedur pada subbab \ref{sec:Lain-lain}.

Fungsi ini menerima tujuh integer sebagai masukan dimana lima pertama merupakan masukan soal, dan dua selanjutnya merupakan \textit{signature} yang dibuat. Keluaran fungsi ini adalah enumerasi benar\textendash salah. Desain fungsi ini dapat dilihat pada gambar \ref{psdo:verify}.
\begin{figure}[h!]
\begin{lstlisting}[firstnumber=0]
VERIFY (q, p, g, y, Hm, r, s)
let w = MODULAR-INVERSE (s, q)
let u1 = MODULAR-MULTIPLICATION (w, Hm, q)
let u2 = MODULAR-MULTIPLICATION (w, r, q)
let g' = MODULAR-EXPONENTIATION (g, u1, p, p-1)
let y' = MODULAR-EXPONENTIATION (y, u2, p, p-1)
let v = MODULAR-MULTIPLICATION (g’, y’, q)
if v == r
	return "SIGNATURE CORRECT"
return "SIGNATURE INCORRECT"
\end{lstlisting}
\caption{Desain Fungsi Uji Kebenaran}
\label{psdo:verify}
\end{figure}

\section{Desain Program Pembuat Data Uji}

Program tahapan komputasi utama memiliki masukan yang tidak sembarangan. Masukan program memiliki sifat-sifat yang dijelaskan pada subbab \ref{sec:Parameter Masukan}. Program ini membentuk data uji untuk membantu memeriksa kebenaran program utama.

\subsection{Desain Fungsi Main Program Pembuat Data Uji}

Semua hal yang berkaitan dengan pembuatan data uji berjalan pada fungsi ini. Desain fungsi ini dapat dilihat pada gambar \ref{psdo:main_generator}.

\begin{figure}[h!]
\begin{lstlisting}[firstnumber=0]
MAIN
let begin = Input()
let end = Input()
let bitDifference = Input()
let limit = Input()
let Hm = Input()
for i = begin to end
	GENERATE (i, i + bitDifference, Hm, limit)
\end{lstlisting}
\caption{Desain Fungsi Utama Pembuat Data Uji}
\label{psdo:main_generator}
\end{figure}

\subsection{Desain Fungsi Pengecekan Keprimaan}

Subbab ini akan menjelaskan metode pengecekan keprimaan berdasarkan penjelasan pada subbab \ref{sec:Miller-Rabin Primality Test}. Masukan fungsi ini adalah sebuah bilangan \textit{integer}. Keluaran fungsi ini adalah enumerasi prima\textendash komposit. Desain fungsi ini dapat dilihat pada gambar \ref{psdo:miller_rabin}.
\begin{figure}[h!]
\begin{lstlisting}[firstnumber=0]
IS-PRIME (value)
if value == 2
	return "PRIME"
if value < 2 or value is divisible with 2
	return "COMPOSITE"
let s = 0
let r = value – 1
while r is divisible with 2
	s = s + 1
	r = r / 2
let PRIMES be array of 12 first primes
let LIMITS be array of 12 Miller-Rabin upper limits
let limit = first index where value < LIMITS[index]
for i = 0 to limit - 1
	let m = MODULAR-EXPONENTIATION (value, a[i], r, r-1)
	if y != 1 or y != (value – 1)
		let j = 1
		while j < s-1 and y != (value – 1)
			y = (y * y) mod value
			if y == 1
				return "PRIME"
			j = j + 1
		if y != (value – 1)
			return "COMPOSITE"
return "PRIME"
\end{lstlisting}
\caption{Desain Fungsi Pengecekan Keprimaan Miller Rabin}
\label{psdo:miller_rabin}
\end{figure}

\subsection{Desain Fungsi Polinomial Prima}

Fungsi ini mengevaluasi polinomial pada persamaan \eqref{eq:polinomial_pembuat_prima} pada subbab \ref{sec:Polinomial Pembuat Bilangan Prima} untuk membentuk bilangan prima. Fungsi ini menerima sebuah \textit{integer} sebagai masukan, dan mengeluarkan \textit{integer} sebagai keluaran. Desain fungsi ini dapat dilihat pada gambar \ref{psdo:prime_generator_polynomial}.
\begin{figure}[h!]
\begin{lstlisting}[firstnumber=0]
POLY-PRIME (n)
return n*n + n + 41
\end{lstlisting}
\caption{Desain Fungsi Polinomial Pembuat Bilangan Prima}
\label{psdo:prime_generator_polynomial}
\end{figure}

\subsection{Desain Fungsi Generate}

Fungsi ini bertujuan untuk membuat data uji dengan karakteristik masukan yang diberikan. Dalam satu pemanggilan fungsi, akan dibuatkan paling banyak sebanyak \textit{limit} data uji yang memenuhi karakteristik masukan. Fungsi ini menerima empat \textit{integer} sebagai masukan, dan delapan \textit{integer} sebagai keluaran. Tujuh dari delapan \textit{integer} pertama merupakan nilai masukan soal mengikuti penjelasan di subbab \ref{sec:Parameter Masukan}. Desain fungsi ini dapat dilihat pada gambar \ref{psdo:generate_1} dan \ref{psdo:generate_2}.
\begin{figure}[h!]
\begin{lstlisting}[firstnumber=0]
GENERATE (N, L, Hm, limit)
let NLimit be the highest N-bit value possible
let LLimit be the highest L-bit value possible
let NBegin be the highest (N-1)-bit value possible
let LBegin be the highest (L-1)-bit value possible
let q = 0
let i = sqrt (NBegin)
while q < NLimit
	q = POLY-PRIME (i)
	i = i + 1
	if IS-PRIME (q) == "COMPOSITE" or q < NBegin
		continue
	for p = LBegin + q – (LBegin mod q) to LLimit
		if IS-PRIME (p) == "COMPOSITE" or p is not divisible with q
			continue
		let g = 1
		for k = 2 to q-1
		find a value of g = k^(p/q) mod (p+1) that doesn’t result in 1 and g^q mod (p+1) results in 1
\end{lstlisting}
\caption{Desain Fungsi \textit{Generate (bg. 1)}}
\label{psdo:generate_1}
\end{figure}

\begin{figure}[t!]
\begin{lstlisting}[firstnumber=last]
		if such g found
			let x a random value between 1 and q-1 inclusive
			let y = MODULAR-EXPONENTIATION (g, x, j + 1, j)
			output (N, L, q, j + 1, g, y, Hm, x)
			limit = limit - 1
			if limit == 0
				quit function
\end{lstlisting}
\caption{Desain Fungsi \textit{Generate (bg. 2)}}
\label{psdo:generate_2}
\end{figure}