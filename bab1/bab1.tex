\vspace{0ex}
\chapter {PENDAHULUAN}

Pada bab ini, akan dijelaskan mengenai latar belakang, rumusan masalah, Batasan masalah, tujuan, metodologi pengerjaan, dan sistematika penulisan tugas akhir.

\section{Latar Belakang}

\par Pengiriman pesan oleh dua pihak (sebut saja, Alice sebagai pengirim dan Bob sebagai penerima) yang berada di mesin (komputer) yang berbeda memiliki beberapa masalah pada aspek keamanan. Salah satunya adalah mengenai integritas pesan. Tidak ada jaminan bahwa pesan yang diterima Bob merupakan pesan yang sama yang dikirimkan oleh Alice. Selain itu, tidak ada jaminan juga bahwa pesan yang diterima Bob memang benar berasal dari Alice. Kedua masalah ini membuka celah terjadinya tindak kriminal atau tersiarnya informasi yang bersifat rahasia. Kedua masalah ini dapat ditangani dengan menggunakan skema Digital Signature dalam mengirimkan pesan. \cite{stallings_cryptography}

Digital Signature merupakan skema untuk memastikan keotentikan sebuah dokumen dan keotentikan sumber pengiriman \cite{stallings_cryptography}. \textit{Signing} akan dilakukan pada pesan yang akan dikirimkan. Hasil \textit{signing} tersebut akan dikirimkan bersama pesan. Penerima pesan dapat dengan mudah memvalidasi apakah \textit{signature} pesan yang diterima sesuai dengan \textit{signature} yang datang bersama pesan tersebut. Namun bahkan dengan skema Digital Signature, keotentikan pesan yang diterima tidak sepenuhnya terjamin. Ada kemungkinan bahwa pihak ketiga membuat sebuah \textit{signature} berdasarkan pesan palsu, sehingga Bob sebagai penerima pesan melihat bahwa pesan yang diterima seolah-olah datang dari Alice. Sebuah \textit{signature} dikatakan dipalsukan apabila untuk sembarang pesan, sebuah \textit{signature} \textbf{baru} untuk pesan tersebut berhasil dibuat \cite{goldwasser_adaptive_chosen_message}.

Terdapat beberapa jenis skema Digital Signature, seperti ElGamal, Schnorr, dan DSA (Digital Signature Algorithm). Tugas akhir ini berfokus pada skema yang ditetapkan sebagai standar Digital Signature: Digital Signature Algorithm \cite{stallings_cryptography}.

Digital Signature Algorithm menggunakan \textit{public-key cryptosystem}, yaitu penggunaan pasangan kunci publik dan kunci privat. \textit{Public-key cryptosystem} yang digunakan Digital Signature Algorithm menggunakan penjelasan berikut.

Diberikan tiga buah nilai, $ g $, $ x $, dan $ p $ dimana terdapat sebuah nilai $ q $ yang memenuhi persamaan $ g^q \equiv 1\ (mod\ p) $. Hitung nilai $ y $ menggunakan persamaan
\[
	y = g^x\ (mod\ p) \text{ untuk $ g $, $ x $, dan $ p $ tertentu}
\]
Maka \textit{public key} yang terbentuk adalah $\langle p, q, g, y \rangle$ dan private key yang terbentuk adalah $ x $.

Salah satu hal penting untuk menjamin efektifitas skema DSA adalah kunci privat tidak boleh diketahui oleh pihak luar. \textit{Total Break} adalah kejadian saat kunci privat milik seorang pengguna berhasil dicari oleh pihak ketiga \cite{goldwasser_adaptive_chosen_message}. Dengan didapatnya kunci privat ini, pihak ketiga dengan bebas bisa membuat \textit{signature} untuk seluruh pesan yang diinginkan.

Pencarian kunci privat pada skema Digital Signature Algorithm adalah pencarian nilai $ x $ pada persamaan $ y = g^x\ (mod\ p) $. Permasalahan ini umum disebut dengan permasalahan logaritma diskret. Permasalahan ini merupakan permasalahan yang sulit untuk diselesaikan \cite{stallings_cryptography}. Meskipun begitu, Pollard dan Shank masing-masing mengajukan satu metode yang dapat digunakan untuk menyelesaikan permasalahan logaritma diskret dengan kompleksitas yang sama, yaitu $ O(n) $ \cite{hac_numtheory}. Brent kemudian meningkatkan metode yang diajukan Pollard sebesar 24\% \cite{brent_montecarlo}. 

Tugas akhir ini akan membangun dan membandingkan dua metode penyelesaian permasalahan logaritma diskret sebagai upaya melakukan pemalsuan \textit{signature}: Shank's Baby-step Giant-step dan Pollard Rho dengan deteksi siklus Brent. Tugas akhir ini akan menggunakan studi kasus \textit{problem} DSA Attack pada situs Timus Online Judge, dimana parameter skema Digital Signature Algorithm yang digunakan lebih sederhana:

\begin{enumerate}
	\item $ N $, bilangan bulat yang menentukan panjang bit parameter $ q $.
	\item $ L $, bilangan bulat yang menentukan panjang bit parameter $ p $.
	\item $ q $, sebuah komponen \textit{public key} berupa bilangan prima.
	\item $ p $, sebuah komponen \textit{public key} berupa bilangan prima dimana $ p-1 $ habis dibagi $ q $.
	\item $ g $, sebuah komponen \textit{public key} berupa bilangan dengan \textit{multiplicative order modulo} $ p $ sebesar $ q $.
	\item $ y $, sebuah komponen \textit{public key} bilangan yang terbentuk dari $ y = g^x\ (mod\ p) $ untuk $ x $ tertentu.
	\item $ H(m) $, sebuah pesan yang telah di-\textit{hash}.
\end{enumerate}

\section {Rumusan Masalah}

Permasalahan yang akan diselesaikan pada tugas akhir ini adalah sebagai berikut:

\begin {enumerate}
\item Bagaimana performa metode \textit{Baby-step Giant-step} yang diimplementasikan pada saat menyelesaikan permasalahan DSA Attack?
\item Bagaimana performa metode \textit{Pollard Rho} yang diimplementasikan pada saat menyelesaikan permasalahan DSA Attack?
\end {enumerate}

\section {Batasan Masalah}

Masalah yang akan diselesaikan memiliki batasan-batasan berikut:

\begin {enumerate}
\item Implementasi dilakukan menggunakan bahasa pemrograman C++.
\item Nilai \textit{hash} pesan memiliki panjang paling sedikit 3 bit, dan paling banyak 36 bit. 
\item Parameter \textit{public key q} memiliki panjang paling sedikit 3 bit, dan paling banyak 36 bit. Panjang parameter \textit{q} sama dengan panjang nilai \textit{hash} pesan.
\item Parameter \textit{public key p} memiliki panjang paling sedikit 6 bit, dan paling banyak 60 bit. Panjang \textit{p} harus setidaknya 3 bit lebih panjang daripada \textit{q}.
\item Parameter \textit{public key g} memiliki rentang nilai $1 < g < p$.
\item Parameter \textit{public key y} memiliki rentang nilai $0 \leq y < p$.
\end {enumerate}

\section {Tujuan}

Tujuan tugas akhir ini adalah sebagai berikut:

\begin{enumerate}
\item Mengevaluasi performa metode \textit{Baby-step Giant-step} yang diimplementasikan untuk menyelesaikan permasalahan DSA Attack.
\item Mengevaluasi performa metode \textit{Pollard Rho} yang diimplementasikan untuk menyelesaikan permasalahan DSA Attack.
\end{enumerate}

\section {Metodologi}

Metodologi pengerjaan yang digunakan pada tugas akhir ini memiliki beberapa tahapan. Tahapan-tahapan tersebut yaitu:

\begin{enumerate}
\item Penyusunan proposal\\
Pada tahapan ini penulis memberikan penjelasan mengenai apa yang penulis akan lakukan dan mengapa tugas akhir ini dilakukan. Penjelasan tersebut dituliskan dalam bentuk proposal tugas akhir.
\item Studi literatur\\
Pada tahapan ini penulis mengumpulkan referensi yang diperlukan guna mendukung pengerjaan tugas akhir. Referensi yang digunakan dapat berupa hasil penelitian yang sudah pernah dilakukan, buku, artikel internet, atau sumber lain yang bisa dipertanggungjawabkan.
\item Implementasi algoritma\\
Pada tahapan ini penulis mulai mengembangkan algoritma yang digunakan untuk menyelesaiakan permasalahan DSA Attack.
\item Pengujian dan evaluasi\\
Pada tahapan ini penulis menguji performa algoritma yang digunakan. Hasil pengujian kemudian dievaluasi untuk kemudian dipertimbangkan apakah algoritma masih bisa ditingkatkan lagi atau tidak.
\item Penyusunan buku\\
Pada tahapan ini penulis menyusun hasil pengerjaan tugas akhir mengikuti format penulisan tugas akhir.
\end{enumerate}

\section {Sistematika Penulisan}

Sistematika laporan tugas akhir yang akan digunakan adalah sebagai berikut:

\begin{enumerate}
\item Bab 1. Bagian ini akan menjelaskan mengenai konteks tugas akhir yang akan dikerjakan, termasuk latar belakang, tujuan, rumusan masalah, dan metodologi.
\item Bab 2. Bagian ini akan menjelaskan mengenai dasar teori yang akan digunakan dalam pengerjaan tugas akhir.
\item Bab 3. Bagian ini akan menjelaskan mengenai desain program yang akan dibangun berdasarkan dasar teori pada bab 2.
\item Bab 4. Bagian ini akan menjelaskan implementasi desain program yang dijelaskan pada bab 3.
\item Bab 5. Bagian ini akan menjelaskan mengenai hasil pengujian program yang telah diimplementasikan pada bab 4.
\item Bab 6. Bagian ini berisi kesimpulan yang akan menjawab rumusan masalah.
\end{enumerate}