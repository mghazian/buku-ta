\chapter {KATA PENGANTAR}

Puji syukur penulis panjatkan kepada Tuhan Yang Maha Esa. Atas rahmat dan kasih sayangNya, penulis dapat menyelesaikan tugas akhir dan laporan akhir dalam bentuk buku ini.

Pengerjaan buku ini penulis tujukan untuk mengeksplorasi lebih mendalam topik-topik yang tidak diwadahi oleh kampus, namun banyak menarik perhatian penulis. Selain itu besar harapan penulis bahwa pengerjaan tugas akhir sekaligus pengerjaan buku ini dapat menjadi batu loncatan penulis dalam menimba ilmu yang bermanfaat.

Penulis ingin menyampaikan rasa terima kasih kepada banyak pihak yang telah membimbing, menemani dan membantu penulis selama masa pengerjaan tugas akhir maupun masa studi.

\begin {enumerate}
	\item Ibu Lathifah Ratna Wardhany, selaku ibu penulis yang senantiasa menanyakan "sudah bab berapa?", sehingga memecut penulis untuk segera merampungkan pengerjaan tugas akhir. Juga pihak yang selalu mengingatkan untuk berdzikir saat bekerja agar dimudahkan dalam mengerjakan tugas akhir.
	\item Bapak Priyandoko, selaku ayah penulis yang selalu mengingatkan untuk berkunjung ke sanak saudara di Sidoarjo.
	\item Ibu Ir. Endah Wismawati MT., beserta keluarga, selaku pihak yang begitu banyak membantu penulis dalam mengarungi dunia perkuliahan, baik bantuan moral maupun materi. Pengerjaan buku ini penulis persembahkan sebagai rasa hormat dan terima kasih kepada beliau sekeluarga.
	\item Bapak Rully Soelaiman S.Kom.,M.Kom., selaku pembimbing penulis. Berkat bimbingan beliau, lama pengerjaan tugas akhir ini dapat ditekan dari tujuh bulan menjadi satu minggu. Ucapan terima kasih juga penulis sampaikan atas segala perhatian, didikan, pengajaran, dan nasihat yang telah diberikan oleh beliau selama masa studi penulis.
	\item Bapak Ridho Rahman Hariadi, S.Kom., M.Sc., selaku pembimbing penulis yang telah memberikan arahan semasa pengerjaan tugas akhir.
	\item Arya Putra Kurniawan, selaku teman satu kos yang hebat karena bisa (dan mau) banyak bergaul dengan penulis selama empat tahun, juga selaku pihak yang telah sangat banyak membantu penulis menjaga asupan gizi dengan mengajak makan hampir setiap malam.
	\item Petrus Damianus W., selaku teman seperjuangan di kampus yang telah banyak bertukar pikiran dengan penulis.
	\item Rekan-rekan satu angkatan 2014 mahasiswa Teknik Informatika yang tidak lelah membantu penulis semasa masa studi, juga karena kesabaran mereka yang luar biasa dalam menghadapi kelakuan penulis.
	\item Abdul Majis Hasani, Theo Pratama, dan Prasetyo Nugrohadi yang telah membantu penulis dalam membuat laporan tugas akhir ini.
\end {enumerate}

Penulis menyadari bahwa buku ini jauh dari kata sempurna. Maka dari itu, penulis memohon maaf apabila terdapat salah kata maupun makna pada buku ini. Akhir kata, penulis mempersembahkan buku ini sebagai wujud nyata kontribusi penulis dalam ilmu pengetahuan.

\begin{flushright}
Surabaya, Maret 2018 \\*
\vspace{5em}
Muhammad Ghazian
\end{flushright}