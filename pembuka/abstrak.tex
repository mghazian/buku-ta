\chapter {ABSTRAK}

% ---- Indonesian vers.

\noindent\textbf{\MakeUppercase\judul}
\vspace*{1em}

\begin{tabularx}{\linewidth}{ l l X }
	Nama 			& : & \penulis \\
	NRP 			& :	& \nrplama \\
	Departemen 		& : & \jurusanbaru, \newline \fakultasbaru, ITS \\
	Pembimbing I 	& : & \pembimbingsatu \\
	Pembimbing II 	& : & \pembimbingdua
	\vspace*{1em} 	% HACKY--USE ALTERNATIVE IF POSSIBLE %
\end {tabularx}

\noindent\textbf{Abstrak} \\
\itshape
Permasalahan logaritma diskret merupakan permasalahan yang sulit untuk diselesaikan. Pada dunia kriptografi, hal ini sering dimanfaatkan untuk mengamankan subjek sekuritas. Banyak studi telah dilakukan terkait topik logaritma diskret, dan dari studi tersebut dihasilkan banyak metode untuk menyelesaikan permasalahan tersebut. Tugas akhir ini mengulas dua metode tersebut, yaitu metode Baby-step Giant-step dan Pollard Rho dengan varian Brent Cycle Detection. Melalui pengujian dan studi kasus, didapat hasil bahwa kedua metode memiliki fungsi pertumbuhan yang relatif sama, namun waktu yang dibutuhkan oleh Baby-step Giant-step untuk menyelesaikan permasalahan lebih stabil dibandingkan Pollard Rho.

\vspace*{1em}
\noindent\bfseries Kata Kunci: baby-step giant-step; pollard rho; brent; logaritma diskret; teori bilangan; kriptografi;
\normalfont
\cleardoublepage

% ---- English vers.
\chapter {ABSTRACT}
\noindent\textbf{\MakeUppercase\juduleng}
\vspace*{1em}

\begin{tabularx}{\linewidth}{ l l X }
	Name 			& : & \penulis \\
	Student ID		& :	& \nrplama \\
	Department 		& : & \jurusanbarueng, \newline \fakultasbarueng, ITS \\
	Supervisor I 	& : & \pembimbingsatu \\
	Supervisor II 	& : & \pembimbingdua
	\vspace*{1em} 	% HACKY--USE ALTERNATIVE IF POSSIBLE %
\end {tabularx}
	
\noindent\textbf{Abstract} \\
\itshape
Discrete logarithm problem is a difficult task to solve. In cryptography, the hardness nature of discrete logarithm is utilized as a mean to improve security. A numerous study has been undertaken regarding discrete logarithm problem, and from which, a handful of method has been constructed to solve such problem. This thesis reviews two methods to solve discrete logarithm problem, namely Baby-step Giant-step and Pollard Rho with Brent Cycle Detection variant. According to subsequent testing and case study, it appears that both method have relatively similar growth function. Baby-step Giant-step's running time, however, is much more stable compared to Pollard Rho.

\vspace*{1em}
\noindent\bfseries Keywords: baby-step giant-step; pollard rho; brent; discrete logarithm; number theory; cryptography;
\normalfont
\cleardoublepage